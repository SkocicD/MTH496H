\section{Voronoi Diagrams}

The third and final tool of interest for the crust is the Voronoi diagram.
Similarly to the other tools introduced so far, the Voronoi diagram takes in a set of points $S$and returns a structure with useful properties.
But instead of connecting the points in a specific way, the Voronoi diagram draws lines between the points to divide the entire plane into \textit{Voronoi}.
Each of these regions contains one point in $S$, say $p$, and is constructed such that any other point is in the region is closer to $p$ than any other point in $S$.
This definition is much more intuitive than that of the Delaunay triangulation, and lends itself to a more obvious set of applications;
It is not hard to imagine that it is useful to know what the closest point is wherever you are in the plane.
The following is a formal definition of the Voronoi region from Devadoss and O'Rourke.


\begin{definition}[Voronoi Region]
	For a point $p$ in set of points $S$, The Voronoi region of p, Vor(p), is such that
	\[ \mathrm{Vor}(p) = \lbrace x\in \mathbb{R} \hspace{4pt} | \hspace{4pt} ||x-p|| \le ||x-q|| \hspace{4pt} \forall q \in S \rbrace \]
\end{definition}

