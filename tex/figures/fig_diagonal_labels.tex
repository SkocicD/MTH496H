\begin{figure}
	\[
		\begin{tikzpicture}[scale=2]
			% Draw the circle
			\draw (0,0) circle (1);

			% Define vertices (TikZ uses degrees)
			\coordinate (A) at (110:1);
			\coordinate (B) at (180:1);
			\coordinate (C) at (275:1);
			\coordinate (D) at (0:.5);

			% Draw the triangle
			\draw (A) -- (B) -- (C) -- (D) -- cycle;
			\draw (A) -- (C);
			\draw (B) -- (D);

			% Optional: mark the vertices
			\fill (A) circle (1pt);
			\fill (B) circle (1pt);
			\fill (C) circle (1pt);
			\fill (D) circle (1pt);

			% angle labels
			\draw pic["$a_1$", draw, angle radius=4mm,angle eccentricity=1.6]
				{angle = A--C--B};
			\draw pic["$b_1$", draw, angle radius=4mm,angle eccentricity=1.6]
				{angle = A--D--B};

			\draw pic["$a_2$", draw, angle radius=4mm,angle eccentricity=1.6]
				{angle = B--A--C};
			\draw pic["$b_2$", draw, angle radius=4mm,angle eccentricity=1.6]
				{angle = B--D--C};

			\draw pic["$a_3$", draw, angle radius=4mm,angle eccentricity=1.6]
				{angle = C--A--D};
			\draw pic["$b_3$", draw, angle radius=4mm,angle eccentricity=1.6]
				{angle = C--B--D};

			\draw pic["$a_4$", draw, angle radius=4mm,angle eccentricity=1.6]
				{angle = D--C--A};
			\draw pic["$b_4$", draw, angle radius=4mm,angle eccentricity=1.6]
				{angle = D--B--A};


			% Labels
			\node[above] at (A) {$A$};
			\node[left] at (B) {$B$};
			\node[below] at (C) {$C$};
			\node[right] at (D) {$D$};
		\end{tikzpicture}
	\]
	\caption{Angle labels used in \lemref{lem_thin_fat}}
	\label{fig_diagonal_labels}
\end{figure}
